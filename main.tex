\documentclass[a4,10pt,comentarios]{aleph-notas}

% -- Paquetes adicionales
\usepackage{enumitem}
\usepackage{aleph-comandos}
\usepackage{mathrsfs}
\usepackage{booktabs}
\usepackage{array}
\usepackage{longtable}
\usepackage{multicol}

% -- Datos del libro
\institucion{Proyecto Alephsub0}
% \carrera{}
\asignatura{Ecuaciones Diferenciales Ordinarias}
\tema{Formulario: Transformada de Laplace}
\autor{Andrés Merino}
\fecha{Julio 2024}
\fuente{montserrat}

%% --> Logos de las guias
\logouno[4.5cm]{Logos/LogoAlephsub0-02}
\definecolor{colordef}{cmyk}{0.81,0.62,0.00,0.22}

%% --> Comandos adicionales
\newcommand{\lap}[2][s]{\displaystyle \mathscr{L} \left[#2\right](#1)}
\newcommand{\lapi}[2][t]{\displaystyle \mathscr{L}^{-1} \left[#2\right](#1)}

\newcommand{\Gam}[1]{\displaystyle\Gamma\left(#1\right)}
\newcommand{\intd}[4]{\displaystyle\int^{#2}_{#1} #3\ d#4}

\begin{document}

\encabezado

\section{Definiciones}

\begin{center}
    \begin{tabular}{ll}
    \toprule
        \textbf{Función} & \textbf{Definición} \\
    \midrule
        Transformada de Laplace & 
            $\displaystyle \lap{f(t)} = \int_0^{+\infty} e^{-st} f(t) \, dt$.
        \\[6mm]
        Función Gamma & 
            $\displaystyle \Gamma(x) = \int_0^{+\infty} e^{-t} t^{x-1} \, dt$. 
        \\[6mm]
        Función Beta & 
            $\displaystyle \beta(x, y) = \int_0^{1} u^{x-1} (1-u)^{y-1} \, du$. 
        \\[6mm]
        Función de Heaviside & 
            $\displaystyle u(t-a) = 
            \begin{cases}
            1 & \text{si } t > a, \\
            0 & \text{si } t < a.
            \end{cases}$ 
        \\[6mm]
        Convolución & 
            $\displaystyle f(t) \ast g(t) = \int_0^{t} f(u) g(t-u) \, du$. 
        \\[6mm]
        Delta de Dirac$^*$& 
            $\displaystyle \delta(t-a) = 
            \begin{cases}
            +\infty & \text{si } t = a, \\
            0 & \text{si } t \ne a.
            \end{cases}$ 
        \\
    \bottomrule
    \end{tabular}

\vspace{3mm}
{\footnotesize $^*$ De manera formal, esta no es una función, y lo presentado sería solo la idea intuitiva.}
\end{center}


\section{Principales transformadas}

\begin{center}
\begin{tabular}{@{}ll@{}}
    \toprule
        \textbf{Función} & \textbf{Transformada de Laplace} \\
         $f(t)$ & $\lap{f(t)}$ \\
    \midrule
        $1$
        & $\displaystyle \frac{1}{s}, \quad s>0$.
        \\[6mm]
        $t$
        & $\displaystyle \frac{1}{s^2}, \quad s>0$.
        \\[6mm]
        $t^n$ 
        & $\displaystyle \frac{n!}{s^{n+1}}, \quad n \in \mathbb{N},\ s>0$. 
        \\[6mm]
        $t^\alpha$ 
        & $\displaystyle \frac{\Gamma(\alpha+1)}{s^{\alpha+1}}, \quad \alpha>-1,\ s>0$. 
        \\[6mm]
        $e^{at}$ 
        & $\displaystyle \frac{1}{s-a}, \quad s>a$. 
        \\[6mm]
        $\sin(at)$ 
        & $\displaystyle \frac{a}{s^2+a^2}, \quad s>0$. 
        \\[6mm]
        $\cos(at)$ 
        & $\displaystyle \frac{s}{s^2+a^2}, \quad s>0$. 
        \\
    \bottomrule
\end{tabular}
%
\hspace{8mm}
%
\begin{tabular}{@{}ll@{}}
    \toprule
        \textbf{Función} & \textbf{Transformada de Laplace} \\
         $f(t)$ & $\lap{f(t)}$ \\
    \midrule
        $\cosh(at)$ 
        & $\displaystyle \frac{s}{s^2-a^2}, \quad s>|a|$. 
        \\[6mm]
        $\senh(at)$ 
        & $\displaystyle \frac{a}{s^2-a^2}, \quad s>|a|$. 
        \\[6mm]
        $\delta(t-a)$ 
        & $\displaystyle e^{-as}, \quad \text{para todo } s$. 
        \\[6mm]
        $u(t-a)$ 
        & $\displaystyle \frac{e^{-as}}{s}, \quad s>0$.
        \\[6mm]
        $\displaystyle\frac{1}{a}\left(1-e^{-at}\right)$
        & $\displaystyle \frac{1}{s(s+a)}, \quad s>0$.
        \\[6mm]
        $\displaystyle\frac{1}{b-a}\left(e^{-at}-e^{-bt}\right)$
        & $\displaystyle \frac{1}{(s+a)(s+b)}, \quad s>0$.
        \\[6mm]
        $\displaystyle\frac{1}{b-a}\left(be^{-bt}-ae^{-at}\right)$
        & $\displaystyle \frac{s}{(s+a)(s+b)}, \quad s>0$.
        \\
    \bottomrule
\end{tabular}
\end{center}

\section{Propiedades de la Transformada de Laplace}
\begin{itemize}[leftmargin=*]
\item {Linealidad:} 
    \begin{align*}
       &\lap{af(t)+bg(t)}=a\lap{f(t)}+b\lap{g(t)},\\[3mm]
       &\lapi{a\varphi(s)+b\psi(s)}=a\lapi{\varphi(s)}+b\lapi{\psi(s)}.
    \end{align*}
\item {Transformada de la derivada:}
    \begin{itemize}[label=$\bullet$]
      \item $\lap{f'(t)}=s\lap{f(t)}-f(0^+)$,\\[-2mm]
      \item $\lap{f''(t)}=s^2 \lap{f(t)}-sf(0^+)-f'(0^+)$,\\[-3mm]
      \item $\lap{f^{(n)}(t)}=s^n \lap{f(t)}-\sum_{i=1}^{n} s^{n-i}f^{(i-1)}(0^+)$.
    \end{itemize}
\item {Transformada de la integral:}
    \begin{center}
    \begin{tabular}{@{}p{7.2cm}p{8.6cm}@{}}
        $\lap{\intd{0}{t}{f(\tau)}{\tau}}=\frac{1}s\lap{f(t)}$,
        &
        $\lapi{\frac 1 s \varphi(s)}=\intd{0}{t}{\lapi{\varphi(s)}}{t}$.
    \end{tabular}
    \end{center}
\item {Traslación en la base:}
    \begin{center}
    \begin{tabular}{@{}p{7.2cm}p{8.6cm}@{}}
       $\lap{e^{at}f(t)}=\lap[s-a]{f(t)}$,
       &
       $\lapi{\varphi(s-a)}=e^{at}\ \lapi{\varphi(s)}$.
    \end{tabular}
    \end{center}
\item {Derivada de la transformada:}
    \begin{center}
    \begin{tabular}{@{}p{7.2cm}p{8.6cm}@{}}
       $\lap{t^{n}f(t)}=(-1)^{n}\ \frac{d^n}{ds^n}\left(\lap{f(t)}\right)$,
       &
       $\lapi{\varphi^{(n)}(s)}=(-1)^{n}\ t^n\ \lapi{\varphi(s)}$.
    \end{tabular}
    \end{center}
\item {Traslación:}
    \begin{center}
    \begin{tabular}{@{}p{7.2cm}p{8.6cm}@{}}
       $\lap{u(t-a)f(t-a)}=e^{-as}\ \lap{f(t)}$,
       &
       $\lapi{e^{-as}\varphi(s)}=u(t-a)\ \lapi[t-a]{\varphi(s)}$.
    \end{tabular}
    \end{center}
\item {Función periódica:} (con periodo $p$)
    \begin{center}
    \begin{tabular}{@{}p{7.2cm}p{8.6cm}@{}}
       $\lap{f(t)}=\frac{\intd{0}{p}{e^{-st}f(t)}{t}}{1-e^{-ps}}$.
       &
    \end{tabular}
    \end{center}
\item {Convolución:}
    \begin{center}
    \begin{tabular}{@{}p{7.2cm}p{8.6cm}@{}}
       $\lap{f(t)\ast g(t)}=\lap{f(t)}\lap{g(t)}$,
       &
       $\lapi{\varphi(s)\psi(s)}=\lapi{\varphi(s)}\ast \lapi{\psi(s)}$.
    \end{tabular}
    \end{center}
\end{itemize}


\section{Otras propiedades}
\subsection{Función Gamma}
\begin{itemize}[leftmargin=*]
\item $\Gam{x+1}=x\ \Gam{x}$,
\item $\Gam{\frac{1}{2}}=\sqrt{\pi}$,
\item $\Gam{n+1}=n!$, para todo $n\in \mathbb{N}$.
\end{itemize}

\subsection{Función Beta}
\begin{itemize}[leftmargin=*]
\item $\displaystyle\beta(x,y)=\frac{\Gam{x}\Gam{y}}{\Gam{x+y}}$.
\end{itemize}

\subsection{Función de Heaviside o función de salto unitario}
\begin{itemize}[leftmargin=*]
\item
    \[
        f(t)=\left\{\begin{array}{l c r}
        f_1(t)&\text{si}&t<t_1,\\[1mm]
        f_2(t)&\text{si}&t_1\leq t<t_2,\\[1mm]
        f_3(t)&\text{si}&t_2\leq t<t_3,\\ \vdots&\vdots
        \end{array}\right.
    \]
    \[
        f(t)=f_1(t)+u(t-t_1)[f_2(t)-f_1(t)]+u(t-t_2)[f_3(t)-f_2(t)]+\cdots.
    \]
\end{itemize}

\subsection{Convolución}
\begin{itemize}[leftmargin=*]
\item $[f(t)\ast g(t)]\ast h(t)=f(t)\ast [g(t)\ast h(t)]$,
\item $f(t)\ast g(t)=g(t)\ast f(t)$.
\end{itemize}

\subsection{Delta de Dirac}
\begin{itemize}[leftmargin=*]
\item $u'(t-a)=\delta(t-a)$,
\item $\displaystyle \intd{a}{b}{f(t)\delta^{(n)}(t-t_0)}{t}=
        \begin{cases}
            f^{(n)}(t_0) &\text{si } t_0\in(a,b),\\
            0       &\text{si }  t_0\not\in(a,b).
        \end{cases}$
\end{itemize}



\end{document} 
